\documentclass[12pt,titlepage]{article}
\usepackage[margin=1in]{geometry}
\usepackage{parskip}

\let\stdsection\section
\renewcommand\section{\clearpage\stdsection}

\usepackage{hyperref}
\hypersetup{
  linktoc=all
}

\begin{document}
  \begin{titlepage}
    \vspace*{\fill}
    \centering

    \textbf{\Huge CS 480 Course Notes} \\ [0.4em]
    \textbf{\Large Introduction to Machine Learning} \\ [1em]
    \textbf{\Large Michael Socha} \\ [1em]
    \textbf{\large University of Waterloo} \\
    \textbf{\large Winter 2019} \\
    \vspace*{\fill}
  \end{titlepage}

  \newpage 

  \pagenumbering{roman}

  \tableofcontents

  \newpage

  \pagenumbering{arabic}

  \section{Course Overview}
    This is an applied introductory machine learning course covering the basics of machine learning algorithms and data analysis.
    Topics covered include:
    \begin{itemize}
      \item Regression analysis
      \item Probabilistic modeling
      \item Support vector machines
      \item Supervised vs unsupervised learning
      \item Reinforcement learning
      \item Neural networks
    \end{itemize}

  \section{Introduction - What is Machine Learning?}
    Machine learning is the field of study of how computers can improve their performance at tasks (i.e. learn) without being explicitly
    programmed to do so. Machine learning can be useful for tasks for which it is difficult to write a step-by-step imperative program.
    Sample applications include optical character recognition, computer vision, and game playing.

    \subsection{Learning Frameworks}
      \begin{itemize}
        \item \textbf{Supervised Learning:} Goal is to learn a function based on its input and output (e.g. determining if an email is spam
          based on a set of emails labeled as spam or not spam).
        \item \textbf{Unsupervised Learning:} Goal is to learn a function based on its input alone (e.g. organizing data into clusters).
        \item \textbf{Reinforcement Learning:} Goal is to learn a sequence of actions that maximize some notion of reward (e.g. learning how
          to control a vehicle to perform some maneuver).
      \end{itemize}

    \subsection{Challenges}
      Some of the challenges facing machine learning today are:
      \begin{itemize}
        \item Dealing with large amounts of data (algorithm complexity and distributed computing become very relevant)
        \item Generating reproducible results
        \item Challenges concerning real-world adoption of work (e.g. human computer interaction, robustness, ethical concerns)
      \end{itemize}

  \section{Decision Trees}
    Decision trees contain questions as nodes and answers as edges, and serve to guide to an answer based on a set of observations.

    \subsection{Predictions Using Decision Trees}
      Consider a data set of employees and their job satisfaction, where we use the data set to predict whether an employee is satisfied with their
      job. An employee can have many features, such as age, salary, seniority, working hours. It is infeasible to build all possible decision trees
      when there are many features due to the exponential growth of the tree. Instead, for predictive purposes, it makes sense to focus on those
      questions that are informative to the prediction. For example, if there is no major difference in job satisfaction based on an employee's age,
      then it is unnecessary to include age in a decision tree, since the answer is not informative for the prediction.

      In general, supervised learning problems have a set of possible inputs $X$, an unknown target function $f: X \rightarrow Y$, and a set of function
      hypotheses $\{h | h: X \rightarrow Y\}$. Supervised learning algorithms accept training examples (a pair $(x, y)$ where $x \in X$, $y \in Y$) and
      output a function hypotheses that approximates $f$. When using decision trees for learning, each decision tree is a function hypothesis.

    \subsection{Encoding Functions in Decision Trees}
      Boolean functions can be fully expressed in decision trees, with one branch for true and another for false. Other function can be approximated as a
      boolean function. For example, instead of a node asking what an employee's salary is, it could ask whether the salary is above a certain amount.

    \subsection{Training and Testing}
      The key idea behind decision tree generation algorithms is to grow a tree until it correctly classifies all training examples. The rough procedure
      followed is:

      \begin{enumerate}
        \item If all training data has the same class, create a leaf node and return.
        \item Else, create the best (i.e. most informative) node on which the split the data.
        \item Split the training set over the above node.
        \item Continue the procedure on each subset of training data generated.
      \end{enumerate}

      \subsubsection{Information Content}
        Criteria for finding the ``best'' split of data can be defined mathematically. If event $E$ occurs with probability $P(E)$, then when $E$ occurs, we
        receive $I(E) = \log_2{\frac{1}{P(E)}}$ bits of information. This can be interpreted as that less likely events yield more information.

      \subsubsection{Entropy}
        An information source $S$ which emits results $s_1, s_2, ..., s_i$ with probabilities $p_1, p_2, ..., p_i$ produces information at
        $H(S) = \sum_i^k p_i I(s_i)$. $H(S)$ is known as the information entropy of S. Information entropy can vary between $0$, which indicates no uncertainty
        (i.e. all members of $S$ in same class), and $1$, which indicates high uncertainty (i.e. equal probability of all classes). The best split of data for
        classification is one that maximally decreases its entropy (a concept known as information gain).

    \subsection{Overfitting}
      A hypothesis $h_1 \in H$ is said to overfit training data if there is some alternative hypothesis $h_2 \in H$ that has a larger error over the training
      data but a smaller error over a larger set of inputs. Overfitting can occur due to errors in a data set or just due to coincidental irregularities
      (especially in a small dataset). Overfitting can be avoided by removing nodes with low information gain, either by stopping decision tree construction early
      or by pruning such nodes after the tree is constructed.

    \subsection{Inductive Bias}
      Inductive bias refers to the assumptions made about the target function to predict future outputs. Common inductive biases for decision trees are:
      \begin{itemize}
        \item Assumption that simplest hypothesis is the best (i.e. Occam's razor).
        \item Decision trees with information gain closer to the root are considered better.
      \end{itemize}

      These are examples of preference bias, which influence the ordering of the hypothesis space. This is distinct from restriction bias, which limits the hypothesis
      space.

    \subsection{Advantages and Disadvantages}
      Decision trees are good for:
      \begin{itemize}
        \item Ease of interpretation
        \item Speed of learning
      \end{itemize}

      Limitations of decision trees include:
      \begin{itemize}
        \item High sensitivity, with tree output changing significantly due to small changes in input.
        \item Not good for learning data sets without axis-orthogonal (i.e. constant in at least 1 dimension) decision boundaries.
      \end{itemize}

  \section{Evaluation of Learning}

    \subsection{Performance Formulation}
      Let $\hat{y}$ be an output generated by a function $f$ approximating some target function. Let $y$ be the corresponding
      output of the target function. A loss function $l(y, \hat{y})$ can be used to measure the accuracy of the approximation
      function $f$. Some common loss functions include:
      \begin{itemize}
        \item Squared Loss: $l(y, \hat{y}) = (y - \hat{y})^2$
        \item Absolute Loss: $l(y, \hat{y}) = |y - \hat{y}|$
        \item Zero/One Loss: $l(y, \hat{y}) = 1_{y \neq \hat{y}}$
      \end{itemize}

      We assume that the data coming from our target function comes from some probability distribution $D$, and that our training data
      is a random sample of $(x, y)$ pairs from $D$. A Bayes Optimal Classifier is a classifier that for any input $x$, returns the
      $y$ most likely to be generated by $D$.

      Based on the available training data, the goal of supervised learning is to find a mapping $f$ from $x$ to $y$ such that
      generalization error $\sum_{(x,y)} D(x,y)l(y,f(x))$ is minimized. However, since $D$ is unknown, we instead estimate the error from
      the average error in our training or test data, which is $\frac{1}{N}\sum_{n=1}^N l(y_n, f(x_n))$.

    \subsection{Common Learning Challenges}
      Some common challenges in learning include:
      \begin{itemize}
        \item Inductive bias of the algorithm being distant from the concept actually being learned.
        \item The data itself having challenging characteristics, such as lots of noise, ambiguity, or missing information.
      \end{itemize}

    \subsection{Training vs Validation vs Test Sets}
      Models are initially built based on a training dataset. Test sets (also known as holdout sets) are then used to estimate the generalization
      error. Validation sets are also used to measure the model's performance, but unlike test sets, validation sets can make changes to the model's
      parameters.

    \subsection{Bias-Variance Decomposition}
      Many machine learning algorithms are based on building a formal model based on the training data (e.g. a decision tree). Models have
      parameters, which are characteristics that can help in classification (e.g. a node in a decision tree). Models may also have hyper-parameters,
      which in turn control other parameters in a model (e.g. max height of decision tree).

      Generalization errors result from a combination of noise, variance, and bias. Bias concerns how well the type of model fits the data. Models
      with high bias pay little attention to training data and suffer from underfitting, while models with low bias may pay too much attention to
      training data and become overfitted. Bias and variance tend to be at odds with one another (high bias typically leads to low variance, and vice
      versa). For example, a decision tree that makes the same prediction for all input has high bias and low variance, while a decision tree trained
      to return a correct prediction for each point of training data will have low bias and likely high variance.

    \subsection{Cross Validation}
      Cross validation is a technique for measuring how well a model generalizes. The idea behind it is to break up a training data set into $K$
      equally sized partitions, and use $K-1$ of the partitions as training data and the remaining partition for testing. This should be repeated
      $K$ times, so that all points of data are at some point used for testing. Higher values of $K$ lower the amount of variance of in the error
      estimation. To avoid training and testing data having a different probability distribution, the data should be shuffled before being split.

    \subsection{Bootstrapping}
      Bootstrapping is an alternative to cross validation where instead of dividing a training data set into partitions, a random sample of points
      (with possible duplicates) is used as training data. The remaining points are then used as testing data, with the goal being similar to
      that of cross validation.

    \subsection{Avoiding Overfitting}
      Overfitting often occurs when training data has many features, which allows for an approximation of the target function with many degrees of
      freedom. Overfitting can be avoided by only considering features with a strong correlation to the output. Cross-validation could be applied
      as follows:
      \begin{enumerate}
        \item Divide training data into $K$ groups at random.
        \item Within each group, find a small set of features with strong correlation to the output. Running this step for each group individually
          instead of for the entire training set further helps avoid overfitting.
        \item Build a classifier using the features and examples from the $K - 1$ groups.
        \item Use the classifier to predict the examples and in group $K$ and measure the error.
        \item Repeat 2-4 to produce an overall cross-validation estimate of the error.
      \end{enumerate}

      Bootstrapping can be applied in a similar way to avoid overfitting.

    \subsection{Performance Evaluation of Classifiers}
      Consider the following terminology for classification problems:
      \begin{itemize}
        \item True positive ($TP$) - Examples of class 1 predicted as class 1
        \item False positive ($FP$) - Examples of class 0 predicted as class 1 (Type 1 Error)
        \item True negative ($TN$) - Examples of class 0 predicted as class 0
        \item False negative ($FN$) - Examples of class 1 predicted as class 0 (Type 2 Error)
      \end{itemize}

      \subsubsection{Accuracy and Error}
        The following formulas can be used to measure accuracy and error:
        $$Accuracy = \frac{TP + TN}{TP + TN + FP + FN}$$
        $$ErrorRate = \frac{FP + FN}{TP + TN + FP + FN}$$

      \subsubsection{Precision and Recall}
        Precision and recall can be measured as follows:
        $$P = \frac{TP}{TP + FP}$$
        $$R = \frac{TP}{TP + FN}$$

        Precision measures the ratio of positive predictions that were correct, while recall measures the ratio of total positive instances that were
        predicted. Similarly to how variance and bias are often at odds with one another, so are precision and recall.

      \subsubsection{F-Measure}
        An F-measure (also known as a F1 score) measures a model's accuracy by taking into account both precision and recall as follows:
        $$F = \frac{2PR}{P + R}$$

        To adjust the relative importance of precision vs recall, a weighted F-measure can be used, which is defined as follows:
        $$F = \frac{(1 + \beta^2)PR}{\beta^2 P + R}$$

        In a standard F-measure, $\beta = 1$, $\beta < 1$ means that precision is valued over recall, while $\beta > 1$ means recall is valued over precision.

      \subsubsection{Sensitivity and Specificity}
        Sensitivity is the same measure as recall. Specificity is a measure of how well a classifier avoids false positives, and is measured as:
        $$Specificity = \frac{TN}{TN + FP}$$

  \section{Instance-Based Learning}

    \subsection{Parametric vs Non-Parametric Methods for Supervised Learning}
      Datasets can be represented as a set of points in a high-dimensional space; a data point with $n$ features $f_1, f_2, ... f_n$ can be represented
      with the feature vector $(f_1, f_2, ..., f_n)$ in n-dimensional space. Parametric methods of supervised learning attempt to model the data using
      these parameters, while non-parametric (also known as instance-based) methods do not.

      \subsubsection{Approximation}
        Parametric methods use parameters to create global approximations. Non-parametric methods instead create approximations based on local data.

      \subsubsection{Efficiency}
        Parametric methods do most of their computation beforehand, and the summarize their results in a set of parameters. Non-parametric methods tend to
        have a shorter training time but a longer query answering time.

    \subsection{K-Nearest Neighbors}
      K-nearest neighbors (KNN) is a common non-parametric method. The idea is to predict the value of a new point based on the values of the $K$ most similar
      (i.e. closest) points.

      \subsubsection{Implementation}
        A common implementation of KNN involves looping through all $N$ points in a training set and computing their distance to some point $x$. Then the $K$ nearest
        points are selected. This process can be sped up by storing the data points in a data structure that helps facilitate distance-based search (e.g. a k-d tree).

      \subsubsection{Distance Function}
        ``Nearby'' means of minimal distance, which is commonly defined by Euclidean distance. Other distance functions $d(x, x')$ can be used, though must meet
        the following conditions:
        \begin{itemize}
          \item $d(x, x') = d(x', x)$ (i.e. symmetric)
          \item $d(x, x) = 0$ (i.e. definite)
          \item $d(a, c) \leq d(a, b) + d(b, c)$ (i.e. triangle inequality holds)
        \end{itemize}

      \subsubsection{Decision Boundaries}
        Decision boundaries define the borders of a single classification of input. These boundaries are formed of sections of straight lines that are equidistant
        to two points of different classes. A highly jagged line is an indicator of overfitting, while a simple line is an indicator of underfitting.

      \subsubsection{Selection of K}
        The selection of the value of $K$ is a bias-variance tradeoff. Low values of $K$ have high variance but low bias, while high values of $K$ have low variance
        but high bias. High-values of $K$ result in smoother decision boundaries, which can be a sign of underfitting, and vice versa.

        Values of $K$ can be algorithmically selected using some of the techniques discussed in the section above (e.g. cross validation). In theory, as the number
        of training examples approaches infinity, the error rate of a 1NN classifier is at worst twice that of the Bayes Optimal Classifier.

      \subsubsection{Pre-Processing}
        Some common forms of pre-processing for KNN include:
        \begin{itemize}
          \item Removing undesirable inputs. Common removal methods are:
            \begin{itemize}
              \item Editing methods, which involve eliminating noisy points of data.
              \item Condensation methods, which involve selecting a subset of data that produces the same or very similar classifications.
            \end{itemize}
          \item Use custom weights for each feature (not all features may be equally relevant for the situation)
        \end{itemize}

      \subsubsection{Distance-Weighted Nearest Neighbor}
        A common problem with KNN is that it can be sensitive to small changes in the training data. One way to mitigate with drawback is to compute a weight for
        each neighbor based on its distance (e.g. through a Gaussian distribution), and this weight determines how much of an influence that point's value has.
        This differs from standard KNN which weighs the values of the $K$ nearest neighbors equally and ignores all other values.

      \subsubsection{High Dimensionality}
        In uniformly distributed high-dimensional spaces, distances between points tend to be roughly equal, since there are so many features that changing a few
        features results in only a small change in distance. However, KNN can still be applied in practice for high-dimensional spaces, since data in high-dimensional
        spaces tends to be concentrated around certain hubs rather than uniformly distributed.

\end{document}
